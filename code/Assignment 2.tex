\documentclass{article}
\usepackage[T1]{fontenc}
\usepackage{ragged2e}
\usepackage{natbib}
\title{Euro Cup Stadiums 2024}
\author{Hassan Bdeir, Youssef Khalil}
\date{10/07/2024}

\begin{document}

\maketitle

\section{Introduction}
\begin{justify}
    This Assignment gave us a small insight on how to work in a team and how to communicate properly to successfully finish the job. Moreover, working in a group without using GitHub would be extremely difficult and time-consuming. In addition, this assignment helped us practice and understand better how to properly code with HTML, CSS, JavaScript, and deepen our knowledge with LaTeX.
\end{justify}

\section{Methods}
\begin{justify}
    In Methods, each language will be written in a separate section explaining the purpose of it and how we used it.
\end{justify}

\begin{itemize}
    \item \textbf{HTML}: The purpose of HTML is to create the backbone of the website, by dividing the head and the body sections, dividing the various sections of the body, and connecting the CSS and JavaScript to it using link and script respectively. We entered the title name in the head section so that the page title would have a label. In the body section, we added five paragraph sentences to explain a bit about the web page, and how to extract some info from the pictures. Furthermore, we divided the body into nine sections, four sections were devised for the images, while the rest were for the anchor tag, form, interaction button, and a fetch button. The web page would look very poor without CSS, therefore we linked it to a CSS file.

    \item \textbf{CSS}: CSS is used to make the website look more appealing. We proceeded with CSS by choosing the grey color for our background color and used CSS to add shadings to our sentences to make it look different, we chose the golden color for the title and black for the rest. As for organizing the pictures and making them responsive we used flexbox to ensure that the pictures would not topple each other. We added shadings to the pictures as well for appearance. A title attribute was added to the images in order to pop up a message when the cursor is paused on top of the image. As for the form and "JavaScript Interactive Button," only minor details were added to them for style.

    \item \textbf{JavaScript}: JavaScript is very important to be added to an HTML file if the developer wants their web page to be interactive. That said, we used JavaScript for two main purposes. The first one was to create an interactive button that would randomly choose a winning team from the names we added to the list by pressing it. The second purpose was to use fetch, and for that reason, we added another button for fetch to be included on the web page, but to do that we created or set up a local server from the terminal and connected it with JavaScript, and "Assignment2.txt" file was added to the folder.

    \item \textbf{XML}: XML has multiple purposes, some of which are interchange between different systems and platforms and data structuring and representation that could be read by humans and computers. We created an XML file that contains the name of the stadiums, cost, year that it was built, capacity, and added a descriptive line to state which teams train there.

    \item \textbf{Git/GitHub}: Git is used to create a local repository on your computer, in order to create various branches for your projects and commit on any changes you have done. Whereas GitHub is used for individuals working in a team to push and pull their work while adding commits so the other individuals could cope with any changes that happened regarding a certain file or folder. We used GitHub to push and pull the work that each of us did while committing on each file before pushing it. GitHub helped us a lot, while keeping our project secured.
\end{itemize}

\section{Conclusion}
\begin{justify}
    We found challenges using GitHub, because using an SSH key and adding it to our computers was very new. Fetching with JavaScript was another obstacle that needed some time to figure out. Moreover, the learning curve in this assignment was tremendous, because it slightly increased and polished our knowledge with the subjects assigned to this task, which is considered very beneficial for us when we need to elaborate further our knowledge in these subjects since we have been properly exposed to the fundamentals.
\end{justify}

\end{document}
